\section{Background and Motivation}

In the current digital landscape, the proliferation of politically biased news content presents substantial challenges for democratic discourse and informed decision-making. The increasing polarization of media sources, combined with algorithmic content curation systems, has created online ecosystems where partipants are exposed to ideologically diverse context, yet are becoming politically polarised \citep{flaxman2016filterbubblesechochambers,kitchens2020echochambersfilterbubbles,Garimella_Smith_Weiss_West_2021, cinelli2021echochambereffectsocialmedia}. As people are generally worse at identifying their own biases \citep{wang2020bias} than those of others, this phenomenon raises concerns about how individuals consume, interpret, and share news information.

The automatic detection and classification of political bias in textual content has emerged as a recent research area within natural language processing and computational social science. Such systems have practical applications in media literacy tools, content recommendation algorithms, fact-checking platforms, and journalistic analysis workflows. However, the computational modeling of political ideology remains a challenging problem due to the inherent subjectivity of bias perception, the complex relationship between explicit content and implicit framing, and the need to distinguish between source-level and article-level ideological markers.

Traditional approaches to bias detection have primarily focused on source-level classification, where entire news outlets or authors are categorized according to their general political orientation \citep{baly2020acl, darwish2020aaai, ronnback2025biasoutlets}. While this coarse-grained approach provides useful insights for media analysis, it fails to capture the significant variation in political perspective that can occur within individual articles from the same source. Article-level classification represents a more nuanced and practically valuable approach, enabling fine-grained analysis of ideological content and reducing the risk of overgeneralization based on source reputation alone.
