
\chapter{Related Work}
\label{sec:relatedwork}

Early approaches in automated ideology classification predominantly formulated the problem as binary classification or focused on identifying extreme partisan viewpoints (hyper-partisan news) \citep{hube2018detecting, semeval2019hyperpartisan, krieger2022daroberta, lin2024inditag}. The literature is distinguished by granularity of analysis: while substantial research has focused on \textit{sentence-level} bias detection \citep{spinde2021mbic,spinde2021babe, menzner2025biasscanner}, identifying specific biased phrases or sentences, these approaches often fail to capture broader contextual or cumulative framing effects across entire articles. Article-level classification approaches, which we adopt in this work, consider the holistic political orientation of texts.

\section{Computational Bias Detection}

The computational detection of political bias in news media has evolved from simple keyword-based approaches to sophisticated neural architectures. Early work in this domain focused primarily on lexical and syntactic features, often employing traditional machine learning methods such as Support Vector Machines and logistic regression \citep{recasens2013weasel}. These approaches, while interpretable, suffered from limited capacity to capture complex semantic relationships and contextual dependencies.

The advent of pre-trained language models has transformed the landscape of bias detection. \citet{baly2020we} introduced the use of BERT for political ideology classification, employing triplet-loss objectives to address the fundamental challenge of source leakage. Their work demonstrated that contrastive learning could encourage models to focus on content-based ideological markers rather than source-identifying patterns. This methodology was subsequently refined by \citet{liu2022politics}, who developed the POLITICS model through large-scale continued pre-training with enhanced contrastive objectives.

Recent work by \citet{ronnback2025biasoutlets} has explored the integration of semantic, syntactic, and metalinguistic features for source-level bias detection, demonstrating the potential value of multi-modal feature representations. However, the systematic integration of such features into article-level classification systems remains underexplored.

\section{Human Annotation and Baseline Studies}

The establishment of human performance baselines represents a critical component of bias detection research, yet comprehensive studies remain limited. The Media Bias Identification Corpus (MBIC) by \citet{spinde2021mbic} utilized crowdworkers to identify biased language, achieving only slight to fair inter-annotator agreement (Fleiss's Kappa $\approx 0.21$). In response to these annotation challenges, the Bias Annotations by Experts (BABE) dataset \citep{spinde2021babe} employed expert annotators specifically trained to recognize media bias, resulting in improved but still modest inter-annotator reliability (Krippendorff's Alpha $\approx 0.39$).

These findings suggest that even expert human annotators face considerable challenges in consistently identifying biased content. Moreover, these prior studies primarily focused on bias detection rather than directional classification. This thesis addresses this gap by introducing an informal layman baseline study for political ideology prediction that explicitly distinguish ideological direction.

\section{Large Language Models in Political Analysis}

The recent emergence of large-scale language models has opened new possibilities for political text analysis. While these models have demonstrated impressive performance across diverse NLP tasks, their application to political bias detection remains understudied. Recent work by \citet{ibrahim2024analyzingpoliticalstancestwitter} has shown promise for political stance detection in social media contexts, but comprehensive evaluation on news article classification has not been conducted.

The potential of prompt engineering and few-shot learning approaches for political classification presents both opportunities and challenges. While these methods can leverage the extensive world knowledge encoded in large language models, they also raise concerns about training data contamination and reproducibility. This thesis provides the first systematic evaluation of modern LLMs on political bias classification, examining both zero-shot and fine-tuning paradigms.


% TODO: - Discussion of ethical considerations and potential societal impact
% TODO: - Comparison table of different bias detection approaches in the literature
% TODO: - Brief overview of the GDELT metadata and its relevance to political analysis
% TODO: - Discussion of the challenges in defining and measuring political bias
% TODO: - Ethical considerations in automated bias detection systems
% TODO: - Comparison with related work in stance detection and sentiment analysis
% TODO: - Discussion of dataset bias and annotation quality issues

