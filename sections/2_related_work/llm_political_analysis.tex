\section{Large Language Models in Political Analysis}

The recent emergence of large-scale language models has opened new possibilities for political text analysis. While these models have demonstrated impressive performance across diverse NLP tasks, their application to political bias detection remains understudied. Recent work by \citet{ibrahim2024analyzingpoliticalstancestwitter} has shown promise for political stance detection in social media contexts, but comprehensive evaluation on news article classification has not been conducted.

The potential of prompt engineering and few-shot learning approaches for political classification presents both opportunities and challenges. While these methods can leverage the extensive world knowledge encoded in large language models, they also raise concerns about training data contamination and reproducibility. This thesis provides the first systematic evaluation of modern LLMs on political bias classification, examining both zero-shot and fine-tuning paradigms.


% TODO: - Discussion of ethical considerations and potential societal impact
% TODO: - Comparison table of different bias detection approaches in the literature
% TODO: - Brief overview of the GDELT metadata and its relevance to political analysis
% TODO: - Discussion of the challenges in defining and measuring political bias
% TODO: - Ethical considerations in automated bias detection systems
% TODO: - Comparison with related work in stance detection and sentiment analysis
% TODO: - Discussion of dataset bias and annotation quality issues
