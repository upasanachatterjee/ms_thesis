\section{Future Research Directions}
\label{sec:future-directions}

Future research should investigate additional metalinguistic dimensions that may provide complementary supervision signals for political bias detection. Promising candidates include:

\begin{itemize}
    \item Syntactic Complexity Measures: Political writing styles may exhibit systematic differences in syntactic complexity, sentence length distribution, and grammatical structure patterns that correlate with ideological orientation.

    \item Discourse Markers and Argumentation Patterns: The use of specific rhetorical devices, evidence types, and argumentation strategies may provide discriminative signals for ideological classification.

    \item Named Entity Coverage Analysis: Systematic differences in which political figures, organizations, or geographic locations receive coverage may reflect ideological source preferences.
\end{itemize}

Additionally, advanced multi-task learning architectures could address the limitations of our current static approach through several innovations:

\begin{itemize}
    \item Cross-Task Attention Mechanisms: Explicit modelling of interactions between auxiliary tasks through attention layers could capture more sophisticated metalinguistic interdependencies.

    \item Hierarchical Task Organization: Organizing auxiliary tasks into hierarchical structures that reflect theoretical relationships between metalinguistic dimensions could improve learning efficiency and interpretability.
\end{itemize}
