
\section{Problem Formulation and Technical Approach}
\label{sec:problem-formulation}

Existing approaches to political bias classification rely primarily on semantic content analysis, either through fine-tuned transformer models or specialized contrastive learning frameworks \citep{baly2020we, liu2022politics}. However, political ideology in news articles manifests not only through explicit content but also through metalinguistic markers such as topic selection, framing strategies, and emotional tone. These higher-level discourse features remain largely unexploited in current computational approaches.

We formulate the integration of metalinguistic features as a multi-task learning problem where the primary political bias classification objective is augmented with auxiliary tasks that capture complementary aspects of political discourse:

\begin{enumerate}
    \item \textbf{Primary Task}: Political ideology classification using contrastive triplet-loss objectives
    \item \textbf{Auxiliary Task 1}: Multi-label topic classification capturing thematic focus
    \item \textbf{Auxiliary Task 2}: Regression of sentiment polarity reflecting tonal characteristics
\end{enumerate}

This formulation enables the model to learn representations that are sensitive to both explicit ideological content and implicit stylistic markers that correlate with political orientation.
