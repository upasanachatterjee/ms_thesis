\section{Future Work}
\label{sec:future-work}

Our findings demonstrate that metalinguistic objectives represent a promising research direction, though several methodological constraints limited our investigation and present possible risks for downstream usage. Future work should prioritize three key areas.

Expanding dataset coverage for metalinguistic annotations represents a critical priority. Approximately one-third of our pre-training corpus contained the necessary GDELT metadata, substantially limiting the impact of the auxiliary objectives and potentially overweighting particular ideologies. Future work should focus on expanding metadata coverage and exploring alternative sources.

The efficacy of metalinguistic prediction as a pre-training objective suggests several promising directions for expanding metalinguistic feature sets. Beyond tone, future research could investigate sentiment intensity, subjectivity, emotional valence, and discourse markers as auxiliary objectives. Multi-objective training combining diverse metalinguistic signals may enable models to capture richer representations of ideological framing. Additionally, examining the transferability of these pre-training objectives across different transformer architectures and model sizes would provide insights into their general applicability.

Cross-domain evaluation represents another important research direction. While our approach demonstrated improvements on the AllSides dataset, assessing robustness across diverse political contexts, emerging topics, and out-of-domain applications remains essential. Furthermore, integrating explainability techniques with metalinguistic pre-training could enhance model interpretability, enabling researchers and practitioners to understand how specific linguistic cues influence ideology predictions and potentially identifying systematic biases in the models themselves.

\section{Limitations}
\label{sec:limitations}

Despite promising results, several limitations constrain the generalizability of our findings. The effectiveness of our metalinguistic pre-training approach was substantially constrained by data availability. Of the 4.5 million articles in our \textsc{BigNewsBLN} pre-training corpus, only 1.6 million contained the requisite GDELT \textsc{V2Themes} and \textsc{V2Tone} metadata. This limitation arose as many of the articles in the dataset were published before the GDELT Project began collecting and annotating articles. This data sparsity likely diminished the potential impact of our multi-objective training approach, as the model had fewer opportunities to learn the proposed metalinguistic representations.

Our reliance on the AllSides dataset also limits the scope of evaluation to a single annotation scheme and political context. While AllSides provides high-quality article-level annotations, validation across multiple datasets with different ideological frameworks would strengthen claims about the general effectiveness of metalinguistic objectives.

Additionally, our focus on U.S. news sources constrains applicability to other national political systems, running the risk of creating a classifier that under or overweights particular ideological cues and does not generalise well. Political discourse patterns, media landscapes, and ideological structures vary significantly across countries, potentially limiting the transferability of our approach to non-U.S. contexts. The left-center-right framework that underlies our classification scheme may not adequately capture political dimensions in multi-party systems or contexts with different ideological alignments. 

Our investigation is also limited to English-language content, precluding assessment of cross-linguistic generalization. Metalinguistic features may manifest differently across languages due to varying syntactic structures, cultural communication patterns, and language-specific bias markers. The effectiveness of metalinguistic objectives observed in English may not transfer directly to other linguistic contexts.

While our approach shows promise for U.S. English-language political ideology classification, broader validation across diverse linguistic, cultural, and political contexts is necessary to establish the generalisability of metalinguistic pre-training objectives for political ideology detection.

